
%\newglossary[glg]{cudaglossary}{gcu}{gtn}{Glossary} 


%\makenoidxglossaries

\newglossaryentry{rowmajor}
{
	%type=cudaglossary,
	name={row major},
	description={Row major means that the different rows of an
		matrix is stored sequencial in a 1d array.}
}

\newglossaryentry{coalesces}
{
	%type=cudaglossary,
	name={coalesces},
	description={ Coalesces means that memory is sequencial order (N, N+1, N+2...)}
}

\newglossaryentry{DRAMBursts}
{
	%type=cudaglossary,
	name={DRAM bursts},
	description={DRAM bursts are multiple data transfers from DRAM grouped together. }
}

\newglossaryentry{cornerTuning}
{
	%type=cudaglossary,
	name={corner turning},
	description={Technique used to make efficient memory accesses. }
}

\newglossaryentry{interleavedDataDistribution}
{
	%type=cudaglossary,
	name={interleaved data distribution},
	description={Storing data in separate groups in the memory to use differen banks when accesing the data. }
}

\newglossaryentry{controlFlow}
{
	%type=cudaglossary,
	name={control flow},
	description={The control flow is the program execution path of control structures like loops and branches. }
}

\newglossaryentry{threadDivergence}
{
	%type=cudaglossary,
	name={Thread Divergence},
	description={Threads of a warp may diverge when they follow different control flows. }
}

\newglossaryentry{warpPadding}
{
	%type=cudaglossary,
	name={warp padding},
	description={If the data isn't a multiple of the warp size the last warp will be filled regardlessly with threads. That's called padding of warps. }
}

\newglossaryentry{boundaryConditions}
{
	%type=cudaglossary,
	name={boundary conditions},
	description={If the data isn't a multiple of the warp size there will be some pading threads that would access the wrong data. Boundary conditions prohibits that. }
}

\newglossaryentry{host}
{
	%type=cudaglossary,
	name={host},
	description={ Host always relates to the CPU that lauches the kernels. }
}

\newglossaryentry{device}
{
	%type=cudaglossary,
	name={device},
	description={Device always relates to the GPU that executes the kernels.. }
}

\newacronym{DRAM}{DRAM}{Dynamic Random Access Memory}
\newacronym{GDDRSDRAM}{GDDR SDRAM}{Graphical Double Data Rate - Synchronous Dynamic Random Access Memory}
\newacronym{CUDA}{CUDA}{Compute Unified Device Architecture}
\newacronym{SIMD}{SIMD}{Single Instruction Multiple Data}
\newacronym{GPU}{GPU}{Graphical Processing Unit}
\newacronym{CPU}{CPU}{Central Processing Unit}
\newacronym{SM}{SM}{Streaming Multiprocessor}
\newacronym{PU}{PU}{Processing Unit}
\newacronym{FPU}{FPU}{Floating Point Unit}


