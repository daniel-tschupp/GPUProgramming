% Load Standard Packages:
%---------------------------------------------------------------------------
\usepackage[standard-baselineskips]{cmbright}

\usepackage[ngerman]{babel}						% german hyphenation
\usepackage[utf8]{inputenc}
\usepackage[T1]{fontenc}						% hyphenation of words with
\usepackage{textcomp}							% additional symbols
\usepackage{ae}									% better resolution of Type1-Fonts 
\usepackage{fancyhdr}							% simple manipulation of header and footer 
\usepackage{etoolbox}							% color manipulation of header and footer
\usepackage{graphicx}                      		% integration of images
\usepackage{float}								% floating objects
\usepackage{caption}							% for captions of figures and tables
\usepackage{booktabs}							% package for nicer tables
\usepackage{tocvsec2}							% provides means of controlling the sectional numbering
%---------------------------------------------------------------------------

% Load Math Packages
%---------------------------------------------------------------------------
\usepackage{amsmath}                    	   	% various features to facilitate writing math formulas
\usepackage{amsthm}                       	 	% enhanced version of latex's newtheorem
\usepackage{amsfonts}                      		% set of miscellaneous TeX fonts that augment the standard CM
\usepackage{amssymb}							% mathematical special characters
\usepackage{exscale}							% mathematical size corresponds to textsize
%---------------------------------------------------------------------------

% Package to facilitate placement of boxes at absolute positions
%---------------------------------------------------------------------------
\usepackage[absolute]{textpos}
\setlength{\TPHorizModule}{1mm}
\setlength{\TPVertModule}{1mm}
%---------------------------------------------------------------------------					
			
% Definition of Colors
%---------------------------------------------------------------------------
\RequirePackage{color}                          % Color (not xcolor!)
\definecolor{linkblue}{rgb}{0,0,0.8}            % Standard
\definecolor{darkblue}{rgb}{0,0.08,0.45}        % Dark blue
\definecolor{bfhgrey}{rgb}{0.41,0.49,0.57}      % BFH grey
%\definecolor{linkcolor}{rgb}{0,0,0.8}     		% Blue for the web- and cd-version!
\definecolor{linkcolor}{rgb}{0,0,0}        		% Black for the print-version!
%---------------------------------------------------------------------------

% Code Including
\usepackage{xcolor}
\usepackage{listingsutf8}

% Hyperref Package (Create links in a pdf)
%---------------------------------------------------------------------------
\usepackage[
	pdftex,ngerman,bookmarks,plainpages=false,pdfpagelabels,
	backref = {false},							% No index backreference
	colorlinks = {true},                  		% Color links in a PDF
	hypertexnames = {true},               		% no failures "same page(i)"
	bookmarksopen = {true},               		% opens the bar on the left side
	bookmarksopenlevel = {0},             		% depth of opened bookmarks
	pdftitle = {Seg Choroid},	   		  		% PDF-property
	pdfauthor = {msm6},        			  		% PDF-property
	pdfsubject = {Segmentierungssoftware},		% PDF-property
	linkcolor = {linkcolor},              		% Color of Links
	citecolor = {linkcolor},              		% Color of Cite-Links
	urlcolor = {linkcolor},              		% Color of URLs
]{hyperref}
%---------------------------------------------------------------------------

%---------------------------------------------------------------------------
% Glossary Package
%---------------------------------------------------------------------------

\usepackage[acronym]{glossaries}

\makenoidxglossaries

%\newglossary[glg]{cudaglossary}{gcu}{gtn}{Glossary} 


%\makenoidxglossaries

\newglossaryentry{rowmajor}
{
	%type=cudaglossary,
	name={row major},
	description={Row major means that the different rows of an
		matrix is stored sequencial in a 1d array.}
}

\newglossaryentry{coalesces}
{
	%type=cudaglossary,
	name={coalesces},
	description={ Coalesces means that memory is sequencial order (N, N+1, N+2...)}
}

\newglossaryentry{DRAMBursts}
{
	%type=cudaglossary,
	name={DRAM bursts},
	description={DRAM bursts are multiple data transfers from DRAM grouped together. }
}

\newglossaryentry{cornerTuning}
{
	%type=cudaglossary,
	name={corner turning},
	description={Technique used to make efficient memory accesses. }
}

\newglossaryentry{interleavedDataDistribution}
{
	%type=cudaglossary,
	name={interleaved data distribution},
	description={Storing data in separate groups in the memory to use differen banks when accesing the data. }
}

\newglossaryentry{controlFlow}
{
	%type=cudaglossary,
	name={control flow},
	description={The control flow is the program execution path of control structures like loops and branches. }
}

\newglossaryentry{threadDivergence}
{
	%type=cudaglossary,
	name={Thread Divergence},
	description={Threads of a warp may diverge when they follow different control flows. }
}

\newglossaryentry{warpPadding}
{
	%type=cudaglossary,
	name={warp padding},
	description={If the data isn't a multiple of the warp size the last warp will be filled regardlessly with threads. That's called padding of warps. }
}

\newglossaryentry{boundaryConditions}
{
	%type=cudaglossary,
	name={boundary conditions},
	description={If the data isn't a multiple of the warp size there will be some pading threads that would access the wrong data. Boundary conditions prohibits that. }
}

\newglossaryentry{host}
{
	%type=cudaglossary,
	name={host},
	description={ Host always relates to the CPU that lauches the kernels. }
}

\newglossaryentry{device}
{
	%type=cudaglossary,
	name={device},
	description={Device always relates to the GPU that executes the kernels.. }
}

\newacronym{DRAM}{DRAM}{Dynamic Random Access Memory}
\newacronym{GDDRSDRAM}{GDDR SDRAM}{Graphical Double Data Rate - Synchronous Dynamic Random Access Memory}
\newacronym{CUDA}{CUDA}{Compute Unified Device Architecture}
\newacronym{SIMD}{SIMD}{Single Instruction Multiple Data}
\newacronym{GPU}{GPU}{Graphical Processing Unit}
\newacronym{CPU}{CPU}{Central Processing Unit}
\newacronym{SM}{SM}{Streaming Multiprocessor}
\newacronym{PU}{PU}{Processing Unit}
\newacronym{FPU}{FPU}{Floating Point Unit}





% Set up page dimension
%---------------------------------------------------------------------------
\usepackage{geometry}
\geometry{
	a4paper,
	left=28mm,
	right=15mm,
	top=30mm,
	headheight=20mm,
	headsep=10mm,
	textheight=242mm,
	footskip=15mm
}
%---------------------------------------------------------------------------
% Makeindex Package
%---------------------------------------------------------------------------
\usepackage{makeidx}                         	% To produce index
\makeindex                                    	% Index-Initialisation
%---------------------------------------------------------------------------

